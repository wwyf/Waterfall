\chapter{加分项}

\begin{enumerate}
    \item \textbf{完善的API文档} \\
    项目采用Swagger设计API文档,为每一个请求设计了详细的API说明和交互参数样例,为前后端对接与分离开发带来便利,详情请参考文件中的[附-1-API文档/API文档pdf版.pdf],或也可访问链接\url{https://swagger.ui.wwyf.top/}。
    \item \textbf{采用先进的框架、架构} 
    \begin{enumerate}
        \item \textbf{框架}: 项目前端使用新兴的Vue.js框架,后端使用流行的Flask框架
        \item \textbf{架构}:项目采用了新兴的微服务架构,获得了更高的开发效率与更优秀的项目效果和稳定性。
    \end{enumerate}
    % 框架方面,项目前端使用新兴的Vue.js框架,后端使用流行的Flask框架,在架构方面,。
    \item \textbf{良好的跨平台性} \\
    项目的前端、后端和数据库分别封装成独立的docker容器,该容器由Docker提供跨平台支持,可在任意安装了Docker的平台进行快速部署,无需考虑兼容性和依赖的问题。
    \item \textbf{数据与需求可扩展性} \\
    项目数据库中对每个数据对象单独制表,数据之间的耦合度较低,在需要进行数据扩展的时候会相对便利;同时,在需求可扩展方面,项目后端采用Flask的蓝图系统,对系统的用户、订单和钱包进行单独封装,项目前端采用高度工程化和模块化的Vue.js进行开发,代码耦合度低,在项目需要进行需求扩展时,只需要作为模块添加,并对接数据接口即可。
    \item \textbf{软件安全性}
    \begin{enumerate}
        \item 项目后端采用了参数化查询,避免SQL注入攻击;
        \item 后端所有的接口都进行了严格的基于Session的用户身份校验,避免数据泄露与越权操作;
        \item 全站采用HTTPS协议进行通信,通信过程进行了加密,防止中间人攻击。 
    \end{enumerate}
    \item \textbf{多用户兼容} \\
    项目使用CDN服务进行资源分发,在大量用户访问时能有效地减轻服务器负载,后端采用支持多线程运作的框架,支持多用户访问,同时,在订单分发时我们采用了严格的检测机制,以解决多用户在同时提交订单时产生的冲突问题。
    \item \textbf{服务过程中升级} \\
    如果在服务过程中进行升级,我们可以利用Docker Swarm工具,对系统进行不间断更新业务,Swarm会自动地完成负载均衡工作,将工作负载移到新建立的Docker容器上,从而实现不须停止业务的服务更新,若更新出现问题,还可以方便地回滚。
\end{enumerate}
