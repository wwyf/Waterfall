%%%%%%%%%%%%%%%%%%%%%%% 需求分析 %%%%%%%%%%%%%%%%%%%%%%%%%%%%%%%%%%%%%%%%%%
\chapter{需求分析}
%定义,过去的研究和现在的研究,意义,与图像分割的不同,going deeper
\label{cha:demand_analysis}
\section{问题陈述}

我们将采用C/S架构,建立一个连接微供应商和订购商之间的一个中间平台“鲜天下”。通过该平台,我们得以实现这样的目标:聚合小生厂商的生产力以满足大厂家的订单需求,规模化效应减少运行成本,通过可靠的授信模型为小生厂商提供合适额度的贷款以解决成本问题。

在订购商对虾有较大的需求时,通过“鲜天下”平台,订购商能够在该平台发起订单并支付定金,并实时查看该订单的提交情况与完成情况,如果订单被平台所接受并完成,订购商需要向平台支付剩余款项。

微供应商可以在“鲜天下”平台上注册申请成为认证微供应商,并能够从平台中获得订单。完成订单后,还能够从平台中得到相应款项。同时,根据自己需要,微供应商还可以从“鲜天下”平台申请不超过信用额度的贷款,以此来解决成本问题。

为确保微供应商与订购商之间的协作安全,平台管理人员会对平台所接收到的来自订购商的订单进行审核,并确定是否接受该订单。若接受该订单,平台需要将该订单拆分成多个小订单,并且推送给合适的小生产商在规定时间内完成。同时,平台还需要监控微供应商完成情况,并及时收取货物并完成订购商的订单要求。为了减少微供应商的生产压力,鼓励更多的微供应商参与到该平台中贡献生产力并获得更高的利润,该平台还会记录微供应商相关数据,并根据这些数据,智能地判断每个微供应商可得到的信用额度,以让微供应商获得贷款,减缓它们的资金压力,释放更多的生产力,同时保证平台资金流的安全与稳定。

“鲜天下”平台运行Unix服务器上,能够自动处理用户的请求,并将相关信息存储于数据库中。由于这些信息涉及订单与资金流,该数据库系统必须保证内部数据的一致性,以保证平台的资金流正常运转。


\section{用例析取}

TODO: 需要修改图片,去做申请额度

对“鲜天下”平台,我们完成了用例析取,图示见\autoref{fig:usecase-main}。

%\usepackage{changepage}
%\usepackage{rotating}
%\usepackage{float}
%\usepackage[section]{placeins}
%\begin{sidewaystable}[!Htp]
\begin{figure}[htp]
    %\begin{adjustwidth}{-1.5cm}{-1cm}
    \centering
    \includegraphics[width=17cm]{figure/usecase/uc_main_ver1.png}
    \caption{“鲜天下平台”用例析取}
    \label{fig:usecase-main}
    %\end{adjustwidth}
\end{figure}


\section{用例规约}

\subsection{登录用例的用例规约}

%\usepackage{changepage}
%\usepackage{rotating}
%\usepackage{float}
%\usepackage[section]{placeins}
%\begin{sidewaystable}[!Htp]
    \begin{figure}[htp]
        %\begin{adjustwidth}{-1.5cm}{-1cm}
        \centering
        \includegraphics[width=12cm]{figure/usecase/uc_sub/uc_login.png}
        \caption{登录用例-活动图}
        \label{fig:logon-uml}
        %\end{adjustwidth}
    \end{figure}
    

\begin{enumerate}
	\item \textbf{简要说明}  \\ 本用例描述订购商/微供应商/管理员如何登录到鲜天下平台。
	\item \textbf{参与者} \\ 订购商、微供应商、管理员, 以下简称用户
	\item \textbf{事件流} \\ 相应活动图可见\autoref{fig:logon-uml}。
	\begin{enumerate} 
        \item \textbf{基本事件流} \\ 本用例开始于用户希望登录到鲜天下平台。
        \begin{enumerate}
            \item 系统请求用户输入用户名和密码。
            \item 用户输入用户名和密码。
            \item 系统验证用户输入的用户名和密码。
            \begin{enumerate}
                \item 用户名不存在。
                \item 用户被锁定。
                \item 用户名对应的密码不正确。
            \end{enumerate}
            \item 用户成功登录到主界面并进行其它操作。
        \end{enumerate}
        \item \textbf{后备事件流}
        \begin{enumerate}
            \item 用户名不存在。
            \begin{enumerate}
                \item 系统显示错误信息 “用户名不存在或密码错误, 超过5次后锁定”。
                \item 返回事件流第一步。
            \end{enumerate}
            \item 用户被锁定。
            \begin{enumerate}
                \item 系统显示错误信息 “用户被锁定, 请联系客服申诉处理”。
                \item 返回事件流第一步。
            \end{enumerate}
            \item 用户名对应的密码不正确。
            \begin{enumerate}
                \item 系统显示错误信息 “用户名不存在或密码错误, 超过5次后锁定”。
                \item 系统将用户错误的登录尝试次数 +1。
                \item 检查登录尝试次数是否超过上限, 超过则锁定用户并发送通知短信。
                \item 返回事件流第一步。
            \end{enumerate}
        \end{enumerate}
    \end{enumerate}
    \item \textbf{特殊需求} \\ 密码输入框必须以密文方式呈现。
    \item \textbf{前置条件} \\ 用例开始前, 用户需要打开对应的系统登录界面, 且用户处于未登录状态。
    \item \textbf{后置条件} \\ 如果用例成功, 系统状态转换为登录态;若失败, 系统状态不改变。
\end{enumerate}


\subsection{提交母订单用例的用例规约}



\begin{enumerate}
	\item \textbf{简要说明}  \\ 本用例描述订购商如何向鲜天下平台提交新订单。
	\item \textbf{参与者} \\ 订购商。
	\item \textbf{事件流} \\ 相应活动图可见\autoref{fig:put-order-uml}。
	\begin{enumerate} 
        \item \textbf{基本事件流} \\ 本用例开始于处于登录态的订购商向鲜天下平台申请订购货物。
        \begin{enumerate}
            \item 用户输入需要虾的数量, 配送地点和配送方式。 \\ 
            选择配送地点过程为:系统显示可选地点列表, 用户依次点选省, 市, 区, 系统显示的可选地点列表层级也从省,下降到是市,最后到区。 用户最后需手动输入街道地址。系统会在用户的输入面板上方列出用户使用的过往完整地址,
            用户可以点击历史地址来一次性完成地址输入。 \\ 
            选择配送方式为点选系统提供的几种可选配送方式之一,选择后配送方式其它选项变为不可选状态。
            \item 系统检测用户的输入是否合法。
            \begin{enumerate}
                \item 客户请求的虾数量不在余量范围内。
                \item 客户没有输入完整的地址。
                \item 客户没有选择配送方式。
            \end{enumerate}
            \item 用户选择确认当前信息继续进行预约或取消交易。
            \begin{enumerate}
                \item 用户选择取消。
            \end{enumerate}
            \item 用户选择订金交付方式。 \\ 选项包括: 立即交付定金, 请求人工面谈.
            \begin{itemize}
                \item 若用户选择立即交付定金, 则转入支付页面, 填写支付信息(如稅号和公司抬头等)。
                \item 若用户选择请求人工面谈, 则请求用户填写公司联系人信息并预约会谈时间。
                \item 用户也可选择取消。
            \end{itemize}
            \item 预订结束, 订单被提交
        \end{enumerate}
        \item \textbf{后备事件流}
        \begin{enumerate}
            \item 客户请求的虾数量不在余量范围内。
            \begin{enumerate}
                \item 系统显示错误信息 “您申请的商品余量不足”。
                \item 返回事件流第一步。
            \end{enumerate}
            \item 客户没有输入完整的地址。
            \begin{enumerate}
                \item 系统显示错误信息 “您没有输入完整的地址”。
                \item 返回事件流第一步。
            \end{enumerate}
            \item 客户没有选择配送方式。
            \begin{enumerate}
                \item 系统显示错误信息 “您没有选择配送方式”。
                \item 返回事件流第一步。
            \end{enumerate}
            \item 在确定信息或选择定金交付方式时,用户选择取消。
            \begin{enumerate}
                \item 返回事件流第一步。
            \end{enumerate}
        \end{enumerate}
    \end{enumerate}
    \item \textbf{特殊需求} \\ 无。
    \item \textbf{前置条件} \\ 用例开始前, 用户需要处于已登录状态下, 并打开交易界面。
    \item \textbf{后置条件} \\ 如果用例成功,用户提交的订单被保存为预订态或取消态,预订态的订单需保存在数据库中,取消态的订单直接丢弃,在接受到用户的定金转账后 (无论是立即交付定金/人工面谈后交付定金),订单状态均切换为待完成态。
\end{enumerate}



%\usepackage{changepage}
%\usepackage{rotating}
%\usepackage{float}
%\usepackage[section]{placeins}
%\begin{sidewaystable}[!Htp]
    \begin{figure}[htp]
        %\begin{adjustwidth}{-1.5cm}{-1cm}
        \centering
        \includegraphics[width=0.7\textwidth]{figure/usecase/uc_sub/uc_client_submit_request.png}
        \caption{提交母订单用例-活动图}
        \label{fig:put-order-uml}
        %\end{adjustwidth}
    \end{figure}
    
    

\section{补充规约}

\section{术语表}


